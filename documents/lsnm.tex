\documentclass[12pt]{article}
\usepackage{palatino}
\usepackage{psfig}
\usepackage{subfigure}
\usepackage{fullpage}
\usepackage{natbib}
\usepackage{setspace}

%%% margins all around.
\setlength{\textwidth}{6.5in}
\setlength{\oddsidemargin}{0in}
\setlength{\textheight}{9in}
\setlength{\topmargin}{0in}

\pagestyle{myheadings}
\markboth{Neural network tools - \today}
         {Neural network tools - \today}

\title{Technical description of the programs used for 
       generating neural models} 

\author{Brain Imaging and Modeling Section, NIDCD \\
        National Institutes of Health, 
	Bethesda MD 20892
	}

%---------------------------------------------------------------------
\begin{document}
%---------------------------------------------------------------------

\maketitle

%---------------------------------------------------------------------
\section{Introduction}
%--------------------------------------------------------------------- 
This documentation for implementing large-scale neural models (LSNM)
is based on computer code developed by
\cite{HusHor02} (auditory model) and \cite{TagHor98} (visual model).
The applications are explained first in general terms, followed by
their application to the implementation of \cite{HusHor02}'s auditory
model.

%---------------------------------------------------------------------
\section{Programs}
%--------------------------------------------------------------------- 
Figure~\ref{lsnm} schematizes the process of network generation and
%%%%%%%%%%%%%%%%%%%%%%%%%%%%%%%%%%%%%%%%%%%%%%%%%%%%%%%%%%%%%
\begin{figure}[htp]
  \centerline{\psfig{figure=FIGS/lsnm.eps,width=3in,angle=-90}}
  \caption{Process for generating and executing a network model}
  \label{lsnm}
\end{figure}
%%%%%%%%%%%%%%%%%%%%%%%%%%%%%%%%%%%%%%%%%%%%%%%%%%%%%%%%%%%%%
simulation for a neural network model. 
$au*.out$ and all of the other output files for each module of the
network are generated by executing the command {\bf sim $<au*.s>$},
where $<au*.s>$ contains pointers to the files $au*.rs$ and
$ausimweightlist.txt$. The file $au*.rs$ contains a script of the
experimental timeline for one trial. The file $ausimweightlist.txt$
contains a list of the files containing the connection weights of the
network. The timeline script specifies the stimuli by using the
$*.inp$ files, which contains a description of the stimulus. The list
of files with the network connection weights are in the format
$*.w$. This format is generated by the {\bf netgen $<*.ws>$} command,
where $<*.ws>$ is a file created by hand (or backprop in the visual
model) containing weight values. The script {\bf makeweights} can be
used here in order to execute {\bf netgen} for each one of the
$<*.ws>$ files. 
 
%---------------------------------------------------------------------
\section{Commands}
%--------------------------------------------------------------------- 
Two unix commands are used. $netgen$, to generate the network, and
$sim$, to execute a given simulation.

%---------------------------------------------------------------------
\subsection{netgen}
%--------------------------------------------------------------------- 
\paragraph{SYNOPSIS} netgen $[*.ws]$
\paragraph{DESCRIPTION} For the given $*.ws$ file, it generates a
weights file $*.w$. The script shells can be used here to execute
netgen1 on several files. The script shell is called $mkweights$.

%---------------------------------------------------------------------
\subsection{sim}
%--------------------------------------------------------------------- 
\paragraph{SYNOPSIS} sim $[*.s]$
\paragraph{DESCRIPTION} Given an experimental trial script, it runs
a simulation and generates output files.

%---------------------------------------------------------------------
\section{Generation of network structure with netgen}
%--------------------------------------------------------------------- 
Programs used here are in C++, and they are:
\begin{description}
  \item[unixmain.cc] Main netgen interface
  \item[generate.cc] Generates the network
\end{description}
The following headers are necessary to run the cc programs:
\begin{description}
  \item[win.h] Definitions specific to unix
  \item[defs.h] Definitions of command options
  \item[structs.h] Defines the network structure 
  \item[decls.h] 
  \item[prototyp.h]
\end{description}
A makefile can be used to generate the executable {\bf netgen}:
\begin{description}
  \item[makefile]
\end{description}

%---------------------------------------------------------------------
\section{Executing the simulation with sim}
%--------------------------------------------------------------------- 
Programs used here are in C++, and they are:
\begin{description}
  \item[unixmain.cc] Main {\bf sim} interface
  \item[win.cc] Functions specific to unix
  \item[init.cc] Initializes simulation parameters
  \item[parse.cc] Interprets the commands for simulation
  \item[actfuncs.cc] Neuron activation functions
  \item[messages.cc] Defines error messages
  \item[outfuncs.cc] Functions for computing output of a neuron
  \item[parsecxn.cc] Parses connection commands
  \item[parseset.cc] Defines set structure
  \item[siminit.cc] Initializes data structures
  \item[simlearn.cc] Learns weights with hebbian rule
  \item[simulate.cc] Executes network simulation
  \item[writeout.cc] Writes out output data in MATLAB format,
                     including PET calculations.
\end{description}
The following headers are necessary to run the cc programs:
\begin{description}
  \item[win.h] Definitions specific to unix
  \item[simdefs.h]
  \item[classdef.h]
  \item[externs.h]
  \item[macros.h]
  \item[simproto.h]
  \item[simdecls.h]
\end{description}
A makefile can be used to generate the executable {\bf sim}:
\begin{description}
  \item[makefile]
\end{description}

%---------------------------------------------------------------------
\section{Definition of a network}
%--------------------------------------------------------------------- 
%---------------------------------------------------------------------
\subsection{Commands}
%--------------------------------------------------------------------- 
The following commands are available to design experimental trials:
\begin{description}
  \item[SET] Creates a new network module.
  \item[\#INCLUDE] Reads specifications from external file.
  \item[RUN] Runs simulation for specific number of iterations. 
  \item[CONNECT] Connects one module to the other by using the
                 specified pattern of connections.
\end{description}

%---------------------------------------------------------------------
\subsubsection{Object properties for the command SET}
%--------------------------------------------------------------------- 
\begin{description}
  \item[ACTRULE] {\em ActRule}. Possible values are: Clamp, clamp
  activations to set values; SigAct, non-differential style sigmoid
  rule in which a new activation is just a sigmoid of the difference
  between input and threshold; 
  DiffSig, differential equation-style sigmoid activation which add
  sigmoid of the input-threshold difference to old activation and 
  subtracts from it the product of decay and old activation;
  Lin, linear activation rule, Noise, noisy clamp; Sigm, sigmoidal
  activation; Shift, shifting activation rule which shifts the clamped
  activation by $dx$ every $delta$ iterations (useful for learning
  invariance of position of an object). 
  \item[INPUTRULE] In case of input rules. Not implemented.
  \item[LEARNRULE] {\em LearnRule}. Possible values are: Off, no learning;
  Aff, afferent hebbian; Eff, efferent hebbian. 
  \item[OUTPUTRULE] {\em OutputRule}. Possible values are: C; SumOut.
  \item[NODEACT] {\em Node Activation}. Initialize activation
  values. Possible values: ALL, all nodes; 
  \item[BINACT] Initial activation values, set by input matrix
  \item[PHASE] Initial phase values. Possible values: ALL.
  \item[PARAMETER] {\em Parameters}. Set parameter values. Possible
  values: ALL.
  \item[LEARNPARAM]. Learning rule parameter values. Possible values:
  ALL, learn all parameters.
  \item[SHIFT] Explicitly shift activations
  \item[TOPOLOGY]. {\em Topology}. Specifies the network
  topology. Possible values: LINEAR, linear topology; RECT,
  rectangular topology. 
  \item[WRITEOUT] {\em Write}. Set file output to every i iterations 
  \item[WRITEWEIGHTS] {\em WW}. Set weight writeout output to every i 
  iterations
\end{description}

%---------------------------------------------------------------------
\subsubsection{Object properties for the command \#INCLUDE}
%--------------------------------------------------------------------- 
The only parameter here is the name of the file.

%---------------------------------------------------------------------
\subsubsection{Object properties for the command RUN}
%--------------------------------------------------------------------- 
The only parameter here is the number of iterations.

%---------------------------------------------------------------------
\subsubsection{Object properties for the command CONNECT}
%--------------------------------------------------------------------- 
\begin{description}
  \item[FROM] {\em From}. 
\end{description}

%---------------------------------------------------------------------
\subsection{Data structures}
%--------------------------------------------------------------------- 
The global data structure used in the programming of LSNM is
illustrated in Fig.~\ref{structure}. NodeSet is a set of nodes that
can be called a module. Initially, there is reserved space for up to
{\em max} number of modules. Every module has its properties plus a
pointer to the actual array of nodes that compose that structure. This
nodes are of type NodeStruct and can be of different lengths (although
in the current models this number is the same for all modules). Each
node of the module contains its own specification plus pointers to the
weights to which is connected, thereby affecting other nodes in any
location of the network. WeightStruct is the data structure for a
connection weight between two modules. Every connection weight
contains its own set of specifications plus a pointer to the node to
which it affects. 
%%%%%%%%%%%%%%%%%%%%%%%%%%%%%%%%%%%%%%%%%%%%%%%%%%%%%%%%%%%%%
\begin{figure}[htp]
  \centerline{\psfig{figure=FIGS/structure.eps,width=5in,angle=-90}}
  \caption{Global structure of an LSNM model. }
  \label{structure}
\end{figure}
%%%%%%%%%%%%%%%%%%%%%%%%%%%%%%%%%%%%%%%%%%%%%%%%%%%%%%%%%%%%%

Note: only the relevant functions are defined below.
Figure~\ref{nodeSet} illustrates the structure of a module in the
%%%%%%%%%%%%%%%%%%%%%%%%%%%%%%%%%%%%%%%%%%%%%%%%%%%%%%%%%%%%%
\begin{figure}[htp]
  \centerline{\psfig{figure=FIGS/nodeSet.eps,width=2in,angle=-90}}
  \caption{Data structure for a network module. Pointer values are
  located at the right and non-pointer values are located at the
  left. }
  \label{nodeSet}
\end{figure}
%%%%%%%%%%%%%%%%%%%%%%%%%%%%%%%%%%%%%%%%%%%%%%%%%%%%%%%%%%%%%
network. The pointer values that make up this structure are as
follows: 
\begin{description}
  \item[Sfs] Points to a file
  \item[Wtfs] Points to a weight file
  \item[Params]
  \item[LrnParams]
  \item[Vars]
  \item[InitSet]
  \item[OutputRule]
  \item[InputRule]
  \item[ActRule]
  \item[Update]
  \item[LearnRule]
  \item[Node] Points to the first of the actual nodes of the current 
              structure
\end{description}
The non-pointer values that make up this structure are as 
follows:
\begin{description}
  \item[SetName] Character string containing structure name
  \item[SetIndex]
  \item[WriteOut] How frequency the state of the structure are output
                  to a file
  \item[WriteWts] How frequency the weights of the structure are output
                  to a file
  \item[Topology] Network topology
  \item[XDim] Number of elements along the x axis of matrix
  \item[YDim] Number of elements along the y axis of matrix
  \item[NSetParams]
  \item[NLrnParams]
  \item[NSetVars]
  \item[N\_Nodes] Number of nodes in this structure
\end{description} 
Every node of the structure described above is a structure of the type
{\em NodeStruct}.
Figure~\ref{nodeStruct} illustrates the structure of a node in each
%%%%%%%%%%%%%%%%%%%%%%%%%%%%%%%%%%%%%%%%%%%%%%%%%%%%%%%%%%%%%
\begin{figure}[htp]
  \centerline{\psfig{figure=FIGS/nodeStruct.eps,width=2in,angle=-90}}
  \caption{Data structure for a single node. Pointer values are
  located at the right and non-pointer values are located at the
  left. }
  \label{nodeStruct}
\end{figure}
%%%%%%%%%%%%%%%%%%%%%%%%%%%%%%%%%%%%%%%%%%%%%%%%%%%%%%%%%%%%%
module of the network. The pointer values that make up this structure 
are as follows: 
\begin{description}
  \item[InputPtr]
  \item[AltInputPtr]
  \item[InWt] Points to a weight class
  \item[OutWt]
  \item[AltWt]
\end{description}
The non-pointer values that make up this structure are as 
follows:
\begin{description}
  \item[Index] Index of the current node 
  \item[Act]
  \item[OldAct]
  \item[MaxAct]
  \item[SumAct]
  \item[SumInput]
  \item[SumExInput]
  \item[SumInhInput]
  \item[SumWeight]
  \item[Receptor]
  \item[Phase]
  \item[Output]
  \item[Input]
\end{description}
Figure~\ref{weightStruct} illustrates the structure of each of the
%%%%%%%%%%%%%%%%%%%%%%%%%%%%%%%%%%%%%%%%%%%%%%%%%%%%%%%%%%%%%
\begin{figure}[htp]
  \centerline{\psfig{figure=FIGS/weightStruct.eps,width=2in,angle=-90}}
  \caption{Data structure for a single weight. Pointer values are
  located at the right and non-pointer values are located at the
  left. }
  \label{weightStruct}
\end{figure}
%%%%%%%%%%%%%%%%%%%%%%%%%%%%%%%%%%%%%%%%%%%%%%%%%%%%%%%%%%%%%
weights interconnecting the modules of the network. The pointer values
that make up this structure are as follows:
\begin{description}
  \item[Set]
  \item[DestNode]
\end{description}
The non-pointer values that make up this structure are as 
follows:
\begin{description}
  \item[LearnOn]
  \item[Value]
  \item[WtVar]
\end{description} 

%---------------------------------------------------------------------
\subsection{How to define a network}
%--------------------------------------------------------------------- 
The following example illustrates the use of the command SET. In this
example, the MGN module was defined for
the auditory model (taken from {\em auseq.s}, a script file for the
sequences memory model):
\begin{verbatim} 
set(MGNs,81) {
  Write 5
  Topology: Rect(1,81)
  ActRule: Clamp
  OutputRule: SumOut
  Node Activation { ALL 0.01 }
}
\end{verbatim}
The command {\em set} creates a new structure called {\em MGNs}, 
consisting of 81 nodes, and initializes all of the structures'
elements to zero. The instruction {\em Write 5} indicates that the
state of each of the 81 nodes will be recorded in an output file every
5 iterations. The instruction {\em Topology: Rect(1,81)} sets the 
topology of the network to a vector of one column by 81 rows.

The following example illustrates how to use the commands \#INCLUDE
and RUN (taken from {\em auseq.rs}, a script file for the sequences
memory model):
\begin{verbatim}
#include noinp_au.inp
Run 200 		% <----------- 1000 ms
\end{verbatim}

The following example shows the first few lines of a network
connection specifications
(taken from {\em mgnsea1d.w}, a file containing connection weights
between MGN and the excitatory part of down-selective Ai of the
sequences memory model):
\begin{verbatim}
Connect(mgns, ea1d)  {
  From:  (1, 1)  {
    ([ 1,81]  0.047928)     ([ 1, 1]  0.099059)     |              | 
  }
\end{verbatim}
This example file of specification of connection weights was generated
by the command {\bf netgen}. The command {\bf netgen} takes a {\em
*.ws} file and generates the connection code. The original file of
corresponding to the connections above is {\em mgnsea1d.ws}, which
contains the following lines:
\begin{verbatim}
mgns ea1d SV I(1 81) O(1 81) F(1 3) 0 0.0 Offset: 0 0
0.05:0.003 0.10:0.002 0.00:0.002
\end{verbatim}
{\em Note: strict observation of syntax in these files is important 
since a simple syntax mistake can make the weight generation process
to reverberate indefinitely.}
The first two strings in this file, {\em mgns} and {\em ea1d},
represent the two modules between which the weights are being
defined, that is, MGN and Ai excitatory down-selective module. {\em S} 
indicates that the fanout weights will be specified in
the current weights file. Other options are {\em R} and {\em A}. The R
option allows to specify that random weights will be generated. The A
option allows to specify that absolute weights positions will be
used. The symbol {\em V} after the {\em S} indicates a verbose output.  
The expressions $I(1 81)$, $O(1 81)$, and $F(1 3)$ specify the input
set size, output set size, and fanout size, respectively. The three
groups of number in the second row of the weights file indicates each
of the weights used for each of the connections. In the above example,
a fanout size of 1 to 3 was specified, which means that each unit of
MGNs will be connected to 3 units in EAid. The connections from the
MGN module will made from the current unit to the corresponding unit
in the EAid module and to the 2 nearest neighbors of that unit. The 
weights of those
connections are the numbers in the second row of the
file. Accordingly, $0.05:0.003$ specifies that the first of those
weights will equal $0.05 \pm 0.003$. The second weight, $0.10:0.002$,
represents a connection weight of $0.10 \pm 0.002$. The third weight,
$0.00:0.002$, represents connection weights of $0.00 \pm 0.002$. The
small variations around the specified weight values are made by
addition of bounded random noise. 

%---------------------------------------------------------------------
\section{Simulating an experimental trial}
%--------------------------------------------------------------------- 

%---------------------------------------------------------------------
\subsection{Steps to run the simulation}
%--------------------------------------------------------------------- 
{\bf sim} $<au*>$,
where $<au*>$ is a file with extension *.s, and contains all the
specifications needed to execute one simulation of the network. 

%---------------------------------------------------------------------
\subsection{To plot raster files of the simulations}
%---------------------------------------------------------------------  
run plotoutput.m in MatLab

%---------------------------------------------------------------------
\subsection{To execute the decision module of a simulation}
%---------------------------------------------------------------------  
run decision.m in matlab

Note that this module can be executed independently of the others.
The code counts the number of neurons in R that responded. If this
number exceeds a given threshold, say 5, then it signals a match.
A mismatch is signaled otherwise. 

%---------------------------------------------------------------------
\bibliographystyle{apalike}
\bibliography{lsnm}
%---------------------------------------------------------------------

%---------------------------------------------------------------------
\appendix
%--------------------------------------------------------------------- 

%---------------------------------------------------------------------
\section{The extended auditory model}
%--------------------------------------------------------------------- 
The diagram of the auditory model before modifications is shown in 
Fig.~\ref{model}.
%%%%%%%%%%%%%%%%%%%%%%%%%%%%%%%%%%%%%%%%%%%%%%%%%%%%%%%%%%%%%
\begin{figure}[htp]
  \centerline{\psfig{figure=FIGS/model.eps,width=6in,angle=-90}}
  \caption{Model of prefrontal working memory of auditory 
           stimuli. For simplicity, only main connections are depicted.
           Each module is composed of both an excitatory and
           an inhibitory submodules.}
  \label{model}
\end{figure}
%%%%%%%%%%%%%%%%%%%%%%%%%%%%%%%%%%%%%%%%%%%%%%%%%%%%%%%%%%%%%
This network was extended in order to perform pattern recognition of
long sequences of tonal patterns. The extended network includes three
modules for each prefrontal module in the old network. That is,
the modules FS$_a$ and FS$_b$ were added to FS, D1$_a$ and D1$_b$ were
added to D1, D2$_a$ and D2$_b$ were added to D2, and FR$_a$ and FR$_b$
were added to FR. In order to make this modifications, a set of {\bf
set} instructions were added in the file {\em auseq.s} (see
description of {\bf set} above). 

After adding the new modules in the script file {\em auseq.s},
connection weights were also added by creating {\em *.ws} files in the
{\em weights} directory. The connection weights among the
new modules and between each of the new modules and STG were identical
to the connection weights among the old modules and the old modules
and STG (see below subsections on connection weights between modules 
and submodules). As an example, the names of the new connection 
weights files, 
{\bf estgexfs\_a.w} and {\bf estgexfs\_b.w}, were added to the script
files containing connections weights, namely, the following lines were
added to {\bf ausimweightlist.txt}: 
\begin{verbatim}
#include weights/estgexfs_a.w
#include weights/estgexfs_b.w
\end{verbatim}
The connections internal to the FS module were also added:
\begin{verbatim}
#include weights/exfs_aexfs_a.w
#include weights/exfs_ainfs_a.w
#include weights/infs_aexfs_a.w
#include weights/exfs_bexfs_b.w
#include weights/exfs_binfs_b.w
#include weights/infs_bexfs_b.w
\end{verbatim}
A similar procedure was followed in order to add the connection
weights of the new buffers for modules D1, D2, and R. 

Next, the {\bf *.w} files were regenerated by using the shell script 
{\bf mkweights}, after which the simulation was executed. Remember
that the shell script {\bf mkweights} executes {\bf netgen} for each
of the {\bf *.ws} files:
\begin{verbatim}
#! /bin/csh

foreach file (*.ws)
  ../bin/netgen $file
end
\end{verbatim}

It is assumed that a directory WEIGHTS already exists. The 
WEIGHTS directory contains all the connection weights 
between regions. These weights are contained in *.w files.
The *.w files were generated by executing the command
$netgen1 <*.ws>$
for each of the *.ws files. The shell script $mkweights$
can be used to execute netgen1 for each of those files.
The *.ws files can be modified by hand, unlike most of 
the other input files for network generation and simulation.

The original *.ws files are named in the following way:

%---------------------------------------------------------------------
\subsection{New brain regions}
%--------------------------------------------------------------------- 
A gating mechanism was added, by adding the following structure in the
file {\em auseq.s}:
\begin{verbatim}
set(Gate,1) {
  Write 5 
  ActRule: Clamp
  OutputRule: SumOut
  Node Activation { ALL 1.0 }
}
\end{verbatim}
This command creates a module called {\bf Gate}, writes its state to
an output file every 5 iterations, clamps its activity to the set
activation, and intializes the structure's units to 1. 

This gating mechanism was connected to the FS module in PFC, by adding
a weights file {\em gateinfs.ws} to the weights directory. The
corresponding {\em gateinfs.w} file was generated by {\bf netgen}, and
the following input was added to the file {\em ausimweightlist.txt}:
\begin{verbatim}
#include weights/gateinfs.w
\end{verbatim}
Similar entries were created for the gating signals of the other two
elements of the buffer. 

%---------------------------------------------------------------------
\subsection{Input files}
%--------------------------------------------------------------------- 
\begin{tabular}{ll}
  {\bf auseq.s}             & Contains all the necessary information to 
                              execute \\
  { }                       & an experimental trial \\
  {\bf ausimweightlist.txt} & Description of connections among regions \\ 
  {\bf auseq.rs}            & Timeline of events in an exp. trial \\

  {\bf noinp\_au.inp}       & Useful for a no input stream \\
  {\bf pethiattn.s}         & Something related to attention \\
  {\bf auseq$<n>$.inp}      & Contains the time-varying input for
                              a sound sequence \\  
  {\bf resetall\_au.inp}    & Useful to reset all the nodes
\end{tabular}

%---------------------------------------------------------------------
\subsection{Output files}
%--------------------------------------------------------------------- 
\begin{tabular}{ll}
  {\bf au*.out}    &  General output of the simulation \\
  {\bf debug.txt}  &  Usually empty                    \\
  {\bf mgns.out}   &  MGN units                        \\
  {\bf ea1u.out}   &  Ai up-selective units            \\
  {\bf ea1d.out}   &  Ai down-selective units          \\
  {\bf ea2u.out}   &  Aii up-selective units           \\
  {\bf ea2c.out}   &  Aii contour-selective units      \\
  {\bf ea2d.out}   &  Aii down-selective units         \\
  {\bf estg.out}   &  STG units                        \\
  {\bf efd1.out}   &  D1 units                         \\
  {\bf efd2.out}   &  D2 units                         \\
  {\bf exfr.out}   &  R units                          \\
  {\bf exfs.out}   &  C units                          \\
  {\bf spec\_pet.m} &  MatLab specification of PET      
\end{tabular}

%---------------------------------------------------------------------
\subsection{Connection weights within MGN}
%--------------------------------------------------------------------- 
None

%---------------------------------------------------------------------
\subsection{Connection weights from MGN to Ai}
%--------------------------------------------------------------------- 
\begin{description}
 \item[mgnsea1u.ws] MGN $\rightarrow$ 
                     excitatory Ai up-selective
  \item[mgnsea1d.ws] MGN $\rightarrow$ 
                     excitatory Ai down-selective 
\end{description}

%---------------------------------------------------------------------
\subsection{Connection weights within Ai}
%--------------------------------------------------------------------- 
\begin{description}
  \item[ea1uea1u.ws] Excitatory Ai up-selective $\rightarrow$
                     excitatory Ai up-selective
  \item[ia1uea1u.ws] Inhibitory Ai up-selective $\rightarrow$
                     excitatory Ai up-selective
  \item[ea1uia1u.ws] Excitatory Ai up-selective $\rightarrow$
                     inhibitory Ai up-selective
  \item[ea1dea1d.ws] Excitatory Ai down-selective $\rightarrow$ 
                     excitatory Ai down-selective
  \item[ea1dia1d.ws] Excitatory Ai down-selective $\rightarrow$
                     inhibitory Ai down-selective
  \item[ia1dea1d.ws] Inhibitory Ai down-selective $\rightarrow$
                     excitatory Ai down-selective
\end{description}

%---------------------------------------------------------------------
\subsection{Connection weights from Ai to Aii}
%--------------------------------------------------------------------- 
\begin{description}
  \item[ea1uea2u.ws] Excitatory Ai up-selective $\rightarrow$
                     excitatory Aii up-selective
  \item[ea1uea2c.ws] Excitatory Ai up-selective $\rightarrow$ 
                     excitatory Aii contour-selective
  \item[ea1dea2c.ws] Excitatory Ai down-selective $\rightarrow$ 
                     excitatory Aii contour-selective
  \item[ea1dea2d.ws] Excitatory Ai down-selective $\rightarrow$ 
                     excitatory Aii down-selective
\end{description}

%---------------------------------------------------------------------
\subsection{Connection weights within Aii}
%--------------------------------------------------------------------- 
\begin{description}
  \item[ea2uea2u.ws] Excitatory Aii up-selective $\rightarrow$
                     excitatory Aii up-selective
  \item[ea2uia2u.ws] Excitatory Aii up-selective $\rightarrow$
                     inhibitory Aii up-selective
  \item[ia2uea2u.ws] Inhibitory Aii up-selective $\rightarrow$
                     excitatory Aii up-selective
  \item[ea2cea2c.ws] Excitatory Aii contour-selective $\rightarrow$
                     excitatory Aii contour-selective
  \item[ea2cia2c.ws] Excitatory Aii contour-selective $\rightarrow$
                     inhibitory Aii contour-selective
  \item[ia2cea2c.ws] Inhibitory Aii contour-selective $\rightarrow$
                     excitatory Aii contour-selective
  \item[ea2dea2d.ws] Excitatory Aii down-selective $\rightarrow$
                     excitatory Aii down-selective
  \item[ea2dia2d.ws] Excitatory Aii down-selective $\rightarrow$
                     inhibitory Aii down-selective
  \item[ia2dea2d.ws] Inhibitory Aii down-selective $\rightarrow$
                     excitatory Aii down-selective
\end{description}

%---------------------------------------------------------------------
\subsection{Connection weights from Aii to Ai}
%--------------------------------------------------------------------- 
The following feedback has been deprecated, and is not currently used
in the simulations. 
\begin{description}
  \item[ea2uea1u.ws] Excitatory Aii up-selective $\rightarrow$
                     excitatory Ai up-selective
  \item[ea2dea1d.ws] Excitatory Aii down-selective $\rightarrow$
                     excitatory Aii down-selective
\end{description}
Note that there is not feedback from Aii contour-selective units to
Ai. 

%---------------------------------------------------------------------
\subsection{Connection weights from Aii to STG}
%--------------------------------------------------------------------- 
\begin{description}
  \item[ea2uestg.ws] Excitatory Aii up-selective $\rightarrow$
                     excitatory STG
  \item[ea2cestg.ws] Excitatory Aii contour-selective $\rightarrow$
                     excitatory STG
  \item[ea2destg.ws] Excitatory Aii down-selective $\rightarrow$
                     excitatory STG 
\end{description}

%---------------------------------------------------------------------
\subsection{Connection weights within STG}
%--------------------------------------------------------------------- 
\begin{description}
  \item[estgestg.ws] Excitatory STG $\rightarrow$ excitatory STG
  \item[estgistg.ws] Excitatory STG $\rightarrow$ inhibitory STG
  \item[istgestg.ws] Inhibitory STG $\rightarrow$ excitatory STG
\end{description}

%---------------------------------------------------------------------
\subsection{Connection weights from STG to Aii}
%--------------------------------------------------------------------- 
\begin{description}
  \item[estgea2u.ws] Excitatory STG $\rightarrow$ 
                     excitatory Aii up-selective
  \item[estgea2c.ws] Excitatory STG $\rightarrow$ 
                     excitatory Aii contour-selective 
  \item[estgea2d.ws] Excitatory STG $\rightarrow$
                     excitatory down-selective
\end{description}

%---------------------------------------------------------------------
\subsection{Connection weights from STG to PFC}
%--------------------------------------------------------------------- 
\begin{description}
  \item[estgexfs.ws] Excitatory STG $\rightarrow$ 
                     Excitatory cue-selective (first in buffer).
  \item[estgexfs\_a.ws] Excitatory STG $\rightarrow$ 
                     Excitatory cue-selective (second in buffer). 
  \item[estgexfs\_b.ws] Excitatory STG $\rightarrow$ 
                     Excitatory cue-selective (third in buffer).
\end{description}

%---------------------------------------------------------------------
\subsection{Connection weights within PFC}
%--------------------------------------------------------------------- 
%---------------------------------------------------------------------
\subsubsection{Connection weights between FS and D1}
%--------------------------------------------------------------------- 
\begin{description}
  \item[exfsifd1.ws] Excitatory cue-selective $\rightarrow$
                     inhibitory delay-selective (1st in buffer)
  \item[efd1infs.ws] Excitatory delay-selective $\rightarrow$
                     inhibitory cue-selective (1st in buffer)
  \item[exfs\_aifd1\_a.ws] Excitatory cue-selective $\rightarrow$
                     inhibitory delay-selective (2nd in buffer)
  \item[efd1\_ainfs\_a.ws] Excitatory delay-selective $\rightarrow$
                     inhibitory cue-selective (2nd in buffer)
  \item[exfs\_bifd1\_b.ws] Excitatory cue-selective $\rightarrow$
                     inhibitory delay-selective (3rd in buffer)
  \item[efd1\_binfs\_b.ws] Excitatory delay-selective $\rightarrow$
                     inhibitory cue-selective (3rd in buffer)
\end{description}
%---------------------------------------------------------------------
\subsubsection{Connection weights between FS and D2}
%--------------------------------------------------------------------- 
\begin{description}
  \item[exfsefd2.ws] Excitatory cue-selective $\rightarrow$
                     excitatory cue-and-delay-selective (1st in buffer)
  \item[exfs\_aefd2\_a.ws] Excitatory cue-selective $\rightarrow$
                     excitatory cue-and-delay-selective (2nd in buffer)
  \item[exfs\_befd2\_b.ws] Excitatory cue-selective $\rightarrow$
                     excitatory cue-and-delay-selective (3rd in buffer)
\end{description}

%---------------------------------------------------------------------
\subsubsection{Connection weights between FS and R}
%--------------------------------------------------------------------- 
\begin{description}
  \item[exfsexfr.ws] Excitatory cue-selective $\rightarrow$
                     excitatory response (1st in buffer)
  \item[exfs\_aexfr\_a.ws] Excitatory cue-selective $\rightarrow$
                     excitatory response (2nd in buffer)
  \item[exfs\_bexfr\_b.ws] Excitatory cue-selective $\rightarrow$
                     excitatory response (3rd in buffer)
\end{description}

%---------------------------------------------------------------------
\subsubsection{Connection weights within FS}
%--------------------------------------------------------------------- 
\begin{description}
  \item[exfsexfs.ws] Excitatory cue-selective $\rightarrow$
                     excitatory cue-selective (1st in buffer)
  \item[exfsinfs.ws] Excitatory cue-selective $\rightarrow$
                     inhibitory cue-selective(1st in buffer)
  \item[infsexfs.ws] Inhibitory cue-selective $\rightarrow$
                     excitatory cue-selective(1st in buffer)
  \item[exfs\_aexfs\_a.ws] Excitatory cue-selective $\rightarrow$
                     excitatory cue-selective (2nd in buffer)
  \item[exfs\_ainfs\_a.ws] Excitatory cue-selective $\rightarrow$
                     inhibitory cue-selective (2nd in buffer)
  \item[infs\_aexfs\_a.ws] Inhibitory cue-selective $\rightarrow$
                     excitatory cue-selective (2nd in buffer)
  \item[exfs\_bexfs\_b.ws] Excitatory cue-selective $\rightarrow$
                     excitatory cue-selective (3rd in buffer)
  \item[exfs\_binfs\_b.ws] Excitatory cue-selective $\rightarrow$
                     inhibitory cue-selective (3rd in buffer)
  \item[infs\_bexfs\_b.ws] Inhibitory cue-selective $\rightarrow$
                     excitatory cue-selective (3rd in buffer)
\end{description}

%---------------------------------------------------------------------
\subsubsection{Connection weights between D1 and D2}
%--------------------------------------------------------------------- 
\begin{description}
  \item[efd1efd2.ws] Excitatory delay-selective $\rightarrow$
                     excitatory cue-and-delay-selective (1st in buffer)
  \item[efd2efd1.ws] Excitatory cue-and-delay-selective $\rightarrow$
                     excitatory delay-selective (1st in buffer)
  \item[efd1\_aefd2\_a.ws] Excitatory delay-selective $\rightarrow$
                     excitatory cue-and-delay-selective (2nd in buffer)
  \item[efd2\_aefd1\_a.ws] Excitatory cue-and-delay-selective $\rightarrow$
                     excitatory delay-selective (2nd in buffer)
  \item[efd1\_befd2\_b.ws] Excitatory delay-selective $\rightarrow$
                     excitatory cue-and-delay-selective (3rd in buffer)
  \item[efd2\_befd1\_b.ws] Excitatory cue-and-delay-selective $\rightarrow$
                     excitatory delay-selective (3rd in buffer)
\end{description}

%---------------------------------------------------------------------
\subsubsection{Connection weights between D1 and R}
%--------------------------------------------------------------------- 
\begin{description}
  \item[efd1exfr.ws] Excitatory delay-selective $\rightarrow$
                     excitatory response
  \item[exfrifd1.ws] Excitatory response $\rightarrow$
                     inhibitory delay-selective
  \item[efd1\_aexfr\_a.ws] Excitatory delay-selective $\rightarrow$
                     excitatory response
  \item[exfr\_aifd1\_a.ws] Excitatory response $\rightarrow$
                     inhibitory delay-selective
  \item[efd1\_bexfr\_b.ws] Excitatory delay-selective $\rightarrow$
                     excitatory response
  \item[exfr\_bifd1\_b.ws] Excitatory response $\rightarrow$
                     inhibitory delay-selective
\end{description}

%---------------------------------------------------------------------
\subsubsection{Connection weights within D1}
%--------------------------------------------------------------------- 
\begin{description}
  \item[efd1efd1.ws] Excitatory delay-selective $\rightarrow$
                     excitatory delay-selective
  \item[efd1ifd1.ws] Excitatory delay-selective $\rightarrow$
                     inhibitory delay-selective  
  \item[ifd1efd1.ws] Inhibitory delay-selective $\rightarrow$
                     excitatory delay-selective
  \item[efd1\_aefd1\_a.ws] Excitatory delay-selective $\rightarrow$
                     excitatory delay-selective
  \item[efd1\_aifd1\_a.ws] Excitatory delay-selective $\rightarrow$
                     inhibitory delay-selective  
  \item[ifd1\_aefd1\_a.ws] Inhibitory delay-selective $\rightarrow$
                     excitatory delay-selective
  \item[efd1\_befd1\_b.ws] Excitatory delay-selective $\rightarrow$
                     excitatory delay-selective
  \item[efd1\_bifd1\_b.ws] Excitatory delay-selective $\rightarrow$
                     inhibitory delay-selective  
  \item[ifd1\_befd1\_b.ws] Inhibitory delay-selective $\rightarrow$
                     excitatory delay-selective
\end{description}

%---------------------------------------------------------------------
\subsubsection{Connection weights between D2 and R}
%--------------------------------------------------------------------- 
\begin{description}
  \item[exfrifd2.ws] Excitatory response $\rightarrow$
                     inhibitory cue-and-delay-selective
  \item[exfr\_aifd2\_a.ws] Excitatory response $\rightarrow$
                     inhibitory cue-and-delay-selective
  \item[exfr\_bifd2\_b.ws] Excitatory response $\rightarrow$
                     inhibitory cue-and-delay-selective
\end{description}

%---------------------------------------------------------------------
\subsubsection{Connection weights within D2}
%--------------------------------------------------------------------- 
\begin{description}
  \item[efd2efd2.ws] Excitatory cue-and-delay-selective $\rightarrow$
                     excitatory cue-and-delay-selective
  \item[efd2ifd2.ws] Excitatory cue-and-delay-selective $\rightarrow$
                     inhibitory cue-and-delay-selective
  \item[ifd2efd2.ws] Inhibitory cue-and-delay-selective $\rightarrow$
                     excitatory cue-and-delay-selective
  \item[efd2\_aefd2\_a.ws] Excitatory cue-and-delay-selective $\rightarrow$
                     excitatory cue-and-delay-selective
  \item[efd2\_aifd2\_a.ws] Excitatory cue-and-delay-selective $\rightarrow$
                     inhibitory cue-and-delay-selective
  \item[ifd2\_aefd2\_a.ws] Inhibitory cue-and-delay-selective $\rightarrow$
                     excitatory cue-and-delay-selective
  \item[efd2\_befd2\_b.ws] Excitatory cue-and-delay-selective $\rightarrow$
                     excitatory cue-and-delay-selective
  \item[efd2\_bifd2\_b.ws] Excitatory cue-and-delay-selective $\rightarrow$
                     inhibitory cue-and-delay-selective
  \item[ifd2\_befd2\_b.ws] Inhibitory cue-and-delay-selective $\rightarrow$
                     excitatory cue-and-delay-selective
\end{description}

%---------------------------------------------------------------------
\subsubsection{Connection weights within R}
%--------------------------------------------------------------------- 
\begin{description}
  \item[exfrexfr.ws] Excitatory response $\rightarrow$ 
                     excitatory response
  \item[exfrinfr.ws] Excitatory response $\rightarrow$
                     inhibitory response
  \item[infrexfr.ws] Inhibitory response $\rightarrow$
                     excitatory response
  \item[exfr\_aexfr\_a.ws] Excitatory response $\rightarrow$ 
                     excitatory response
  \item[exfr\_ainfr\_a.ws] Excitatory response $\rightarrow$
                     inhibitory response
  \item[infr\_aexfr\_a.ws] Inhibitory response $\rightarrow$
                     excitatory response
  \item[exfr\_bexfr\_b.ws] Excitatory response $\rightarrow$ 
                     excitatory response
  \item[exfr\_binfr\_b.ws] Excitatory response $\rightarrow$
                     inhibitory response
  \item[infr\_bexfr\_b.ws] Inhibitory response $\rightarrow$
                     excitatory response
\end{description}

%---------------------------------------------------------------------
\subsection{Connection weights from PFC to STG}
%--------------------------------------------------------------------- 
\begin{description}
  \item[efd1istg.ws] Excitatory delay-selective $\rightarrow$
                     inhibitory STG
  \item[efd2estg.ws] Excitatory cue-and-delay-selective $\rightarrow$
                     excitatory STG
  \item[efd1\_aistg.ws] Excitatory delay-selective $\rightarrow$
                     inhibitory STG
  \item[efd2\_aestg.ws] Excitatory cue-and-delay-selective $\rightarrow$
                     excitatory STG
  \item[efd1\_bistg.ws] Excitatory delay-selective $\rightarrow$
                     inhibitory STG
  \item[efd2\_bestg.ws] Excitatory cue-and-delay-selective $\rightarrow$
                     excitatory STG
\end{description}

%---------------------------------------------------------------------
\subsection{Connection weights from PFC to Aii}
%--------------------------------------------------------------------- 
\begin{description}
  \item[efd2ea2u.ws] Excitatory cue-and-delay-selective $\rightarrow$
                     excitatory Aii up-selective
  \item[efd2ea2c.ws] Excitatory cue-and-delay-selective $\rightarrow$
                     excitatory Aii contour-selective
  \item[efd2ea2d.ws] Excitatory cue-and-delay-selective $\rightarrow$
                     excitatory Aii down-selective
  \item[efd2\_aea2u.ws] Excitatory cue-and-delay-selective $\rightarrow$
                     excitatory Aii up-selective
  \item[efd2\_aea2c.ws] Excitatory cue-and-delay-selective $\rightarrow$
                     excitatory Aii contour-selective
  \item[efd2\_aea2d.ws] Excitatory cue-and-delay-selective $\rightarrow$
                     excitatory Aii down-selective
  \item[efd2\_bea2u.ws] Excitatory cue-and-delay-selective $\rightarrow$
                     excitatory Aii up-selective
  \item[efd2\_bea2c.ws] Excitatory cue-and-delay-selective $\rightarrow$
                     excitatory Aii contour-selective
  \item[efd2\_bea2d.ws] Excitatory cue-and-delay-selective $\rightarrow$
                     excitatory Aii down-selective
\end{description}

%---------------------------------------------------------------------
\subsection{Connection weights from Attention module to PFC}
%--------------------------------------------------------------------- 
\begin{description}
  \item[attsefd2.ws]  Attention $\rightarrow$
                      excitatory cue-and-delay-selective
  \item[attsefd2\_a.ws]  Attention $\rightarrow$
                      excitatory cue-and-delay-selective
  \item[attsefd2\_b.ws]  Attention $\rightarrow$
                      excitatory cue-and-delay-selective
  \item[attvatts.ws]  Attention $\rightarrow$ attention         
\end{description}

%---------------------------------------------------------------------
\subsection{Connection weights in control region}
%--------------------------------------------------------------------- 
This region is independent of all the other regions, and is useful as
a control area in order to normalize the activity in the network. 
\begin{description}
  \item[ectlectl.ws] Excitatory control $\rightarrow$
                     excitatory control
  \item[ectlictl.ws] Excitatory control $\rightarrow$
                     inhibitory control
  \item[ictlectl.ws] Inhibitory control $\rightarrow$
                     excitatory control
  \item[ictlictl.ws] Inhibitory control $\rightarrow$
                     inhibitory control
\end{description}

%---------------------------------------------------------------------
\end{document}
%---------------------------------------------------------------------
